\chapter{Thermal Simulations with CoolSPICE}

\label{chap_thermalsimulations_tas}

CoolSPICE now supports two types of thermal effects, for three types of devices. The devices are SiC/GaN MOSFETs, diodes, and resistors. 
For each of these, thermal analysis can be performed to simulate the effect of self-heating, and the effect of thermal coupling. 
Then, the junction temperature or the case (ambient) temperature of the devices may be outputted as traces on the Plotter. \\

In order for thermal simulation to be possible for these devices, the devices must include RTH or CTH device parameters. These parameters may 
be specified as model parameters within a model definition, or as instance parameters, included inline when an instance of one of the devices
is called within the netlist. \\

\begin{tabular}{l l}
\textit{parameter} & \textit{description} \\ \hline \\ \vspace{-0.8\parskip}
\texttt{RTH} & Value of the thermal resistance for MOSFET \\
\texttt{CTH} & Value of the thermal capacitance for MOSFET \\
\texttt{RTHHS} & Value of the thermal resistance for Heat Sink \\
\texttt{CTHHS} & Value of the thermal capacitance for Heat Sink \\
\texttt{RTHD} & Value of the thermal resistance for Diode \\
\texttt{CTHD} & Value of the thermal capacitance for Diode \\
\texttt{RTHDM} & Value of the thermal resistance for Diode and MOSFET thermal coupling \\
\end{tabular}

\section{Netlist Syntax}
\label{subsec_sceadm_thermalnetsyntax}

In order to first enable thermal analysis, the following option card must be included in the netlist.

\spicesyntax{.option heaton} 

Next, when you include the device that you'd like to enable thermal simulation for, the RTH or CTH (or both) instance parameters must be included
inline with the model call. For example for a CMF10120 MOSFET,

\spicesyntax{Mxxx D G S S CMF10120 tamb=27 rd=0.08 rg=10 rth=0.2 cth=5e-5}

Lastly, to save the trace of either the Junction Temperature (TJ) or the Case/Ambient Temperature of a device, all you need to do is use the
device instance parameter syntax (Section \ref{subsec_satco_savestatement}). Then, you're all done! \\








